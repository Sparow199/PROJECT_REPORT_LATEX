\section{Conclusion}

La solution que j'ai apportée correspond aux attentes dans la partie implémentée en langage C.\\

Je suis parti des connaissances acquises durant mon cursus universitaire. Par la suite, j'ai effectué des recherches qui m'ont permis de mettre en place les différentes parties du projet.\\

Les objectifs que je me suis fixés au début du PJI n'ont pas tous été réalisés. J'ai bien réussi la première partie, mais  pour la deuxième, j'ai eu trop de problèmes liés à la fois au matériel et logiciel. J'ai du changer le microcontrôleur, car Arduino Uno r3 n’était pas assez puissant pour supporter le moteur JavaScript et je l'es remplacé par le ESP8266 qui a ses avantages et ses inconvénients.\\

Ce PJI a été très intéressant et instructif. J'ai beaucoup appris sur le monde  de la recherche, ainsi que sur les connaissances qui seront utiles dans mon parcours.
Je suis très attiré par le monde de la recherche et parmi mes objectifs pour le PJI, je voulais me faire faire une idée plus précise de ce milieu. Cette expérience m'a permis de faire des rencontres avec des chercheurs dans le domaine qui m'intéresse. 
Grâce à ce PJI, j'ai pu travailler les connaissances que j'ai étudiées lors de mes cours du master.
C'est grâce à ce projet que j'ai eu une vraie idée sur le travail dans une équipe de recherche, et surtout à devenir autonome sur un projet concret.   
\newpage